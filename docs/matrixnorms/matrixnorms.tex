\documentclass{article}
    % General document formatting
    \usepackage[margin=0.7in]{geometry}
    \usepackage[parfill]{parskip}
    \usepackage[utf8]{inputenc}

    % Related to math
    \usepackage{amsmath,amssymb,amsfonts,amsthm}

\begin{document}

\section*{Why all this?}
Differential privacy is given w.r.t. datasets that "differ in only one entry", but the mechanisms we have for making a function differentially private measure the noise they add in terms of a sensitivity that is given w.r.t. e.g. the $L2$-distance. Hence we need to check sensitivity of functions w.r.t. different input- and output metrics.

\section*{Metric Spaces}
$(M, d)$ where $M$ is a set and $d : M \times M \rightarrow \mathbb{R}$ s.t.
\begin{itemize}
   \item $d(x,y) = 0 \Leftrightarrow x = y$
   \item $d(x,y) = d(y,x)$
   \item $d(x,z) \leq d(x,y) + d(y,z)$
\end{itemize}

\section*{Sensitivity}
for metric spaces $M,N$ a map $f:M\rightarrow N$ is $s$-sensitive if for all $x,y \in M$
$$d_N(f(x), f(y)) \leq s \cdot d_M(x,y)$$
"If the input is at most 1 apart, the output is at most $s$ apart."

\section*{Gaussian Mechanism}
Let $\mathcal{D}$ be some space equipped with the discrete metric (e.g. $(\mathbb{D},L\infty)$). Given a function $f:\mathcal{D}\rightarrow \mathbb{R}^n$ that is $s$-sensitive in $L2$ norm, for every $\delta\in(0,1)$ and $\epsilon\in(0,1)$ the gaussian mechanism
\[\mathcal{M}_\text{Gauss}(f, \epsilon, \delta)(x) = f(x) + \mathcal{N}^n\left(\mu = 0, \sigma^2 = \frac{2 \ln (1.25/\delta) \cdot s^2}{\epsilon^2}\right)\]
yields an $(\epsilon,\delta)$-differentially private function.

\section*{Metrics on Numbers}
\begin{itemize}
   \item On $\mathbb{R}$ the metric is standard:
      $$d_\mathbb{R}(x,y) = |x-y|$$
   \item On $\mathbb{D}$ we just check if the numbers are equal:
      $$d_\mathbb{D}(x,y) = (x==y ? 0 : 1)$$
\end{itemize}


\section*{Metrics on Vectors}
\begin{itemize}
   \item Vectors over $\mathbb{R}$:
      \[  d_{L1,\mathbb{R}}(v,w) = \sum_i d_\mathbb{R}(v_i, w_i) \]
      \[  d_{L2,\mathbb{R}}(v,w) = \sqrt{\sum_i d_\mathbb{R}(v_i, w_i)^2} \]
      \[  d_{L\infty,\mathbb{R}}(v,w) = \max_i d_\mathbb{R}(v_i, w_i) \]
   \item Vectors over $\mathbb{D}$:
      \[  d_{L1,\mathbb{D}}(v,w) = \sum_i d_\mathbb{D}(v_i, w_i) = \textnormal{number of entries that differ} \]
      \[  d_{L2,\mathbb{D}}(v,w) = \sqrt{\sum_i d_\mathbb{D}(v_i, w_i)^2} = \sqrt{d_{L1,\mathbb{D}}(v,w) } \]
      \[  d_{L\infty,\mathbb{D}}(v,w) = \max_i d_\mathbb{D}(v_i, w_i) = \text{"} 0 \text{ if } v=w \text{, 1 otherwise"} \]
   \item some facts
      \begin{itemize}
         \item for vectors $v\neq w$ we have
               \begin{equation}
                   \label{eqn:one}
                    1 = d_{{L\infty},\mathbb{D}}(v,w) \leq d_{L2,\mathbb{D}}(v,w) \leq d_{L1,\mathbb{D}}(v,w)
               \end{equation}
         \item hence if a vector-valued function $f:M->\mathbb{D}^n$ is $s$-sensitive for a fixed input metric and output metric $d_{L1,\mathbb{D}}$, it is also $s$-sensitive under output metric $d_{L\infty,\mathbb{D}}$ because
         \[d_{L\infty,\mathbb{D}}(f(v),f(w))\overset{(\ref{eqn:one})}{\leq} d_{L2,\mathbb{D}}(f(v),f(w)) \overset{(\ref{eqn:one})}{\leq} d_{L1,\mathbb{D}}(f(v),f(w)) \overset{f \textnormal{ is $s$-sensitive}}{\leq} s \cdot d_M(v,w)\]
         \item also all functions from $(*,\mathbb{D})$-vectors to $(L\infty,\mathbb{D})$-vectors are 1-sensitve because for $v \neq w$ it is
         \[1 = d_{L\infty, \mathbb{D}}(v,w) = 1 \overset{(\ref{eqn:one})}{\leq} 1 \cdot d_{L*, \mathbb{D}}(v,w)\]
         \item Clipping $(L\infty,\mathbb{D})$ vectors is such a function and hence 1-sensitive. I don't think clipping vectors in general is 1-sensitive (even if it says so in the paper), because e.g. $d_{L1,\mathbb{D}}$ can become larger:
            $$d_{L1,\mathbb{D}}([1,1],[1,0]) = 1$$
            but for the clipped vectors 
            $$d_{L1,\mathbb{D}}(clip^{L1}([1,1]), clip^{L1}([1,0])) = d_{L1,\mathbb{D}}([0.5,0.5],[1,0]) = 2$$
      \end{itemize}
\end{itemize}

\section*{Matrix Type}
The duet matrix type has the following parameters:
$$\mathbb{M}^c_l \tau[i,j]$$
is the type of matrices where
\begin{itemize}
   \item the matrix has $i$ rows and $j$ columns
   \item all rows have $d_{c,\mathbb{R}}(r,0) \leq 1$ (note that this is the $\mathbb{R}$ norm no matter what $\tau$ is. This differs from what is said on p.44 of the paper, but it makes no sense otherwise and in their implementation it's like we think, see last item of "Implications")
   \item the elements are of type $\tau$ and the metric is chosen accordingly
   \item sensitivities of variables with this type are given w.r.t. $d_{\mathbb{M}^\star_l\tau}$
\end{itemize}

\section*{Metrics over Matrices}
For matrices $m,n \vdash \mathbb{M}^\star_l\tau$ the metric sums over rows:
      $$d_{\mathbb{M}^\star_l\tau}(m,n) = \sum_j d_{l,\tau}(m_j,n_j)$$
In particular,
$$d_{\mathbb{M}^\star_{L1}\mathbb{D}}(m,n) = \text{number of matrix entries that differ}$$
$$d_{\mathbb{M}^\star_{L\infty}\mathbb{D}}(m,n) = \text{number of matrix rows that differ somewhere}$$

\section*{\texttt{discf}}
The function $\texttt{discf} : \mathbb{R}\rightarrow\mathbb{D}$ is claimed to be 1-sensitive in the paper. Taking the numbers $0.1$ and $0.2$ as an example, we get
\[1 = d_{\mathbb{D}}(\texttt{discf}(0.1),\texttt{discf}(0.2)) = 10 \cdot d_{\mathbb{R}}(0.1,0.2)\]
so using our notion of sensitivity \texttt{discf} must be at least 10-sensitive...

I suspect they used a different definition for sensitivity, namely a function $f:M\rightarrow N$ to be $s$-sensitive iff
\[\textnormal{max}_{d_M(x,y)=1} d_N(f(x),f(y)) = s\]
This definition is equivalent to the above one if $M = \mathbb{D}$ but not in general.

\section*{\texttt{convert}}
We can convert $\mathbb{M}^l_\star\mathbb{D}$ to $ \mathbb{M}^l_l\mathbb{R}$ because all rows of the first type have $l$-norm $\leq 1$ so for any two rows $m_i,n_i$ with $d_{\star,\mathbb{D}}(m_i,n_i) \leq 1$ we have 
      $$d_{l,\mathbb{R}}(m_i,n_i) \leq d_{l,\mathbb{R}}(m_i,0) + d_{l,\mathbb{R}}(0,n_i)  = \leq 1 + 1$$
The inequality is met, e.g. by the vectors $[1,0]$ and $[-1,0]$, as they have $\mathbb{D}$-distance 1 and $L2$-norm 1, but $L2$-distance 2 from each other.

This implies conversion of the rows is 2-sensitive, but in the paper it is declared 1-sensitive. Also the paper version does not preserve clipping on the matrix, even though in their interpreter the \texttt{convert} function is simply the identity.

\end{document}
